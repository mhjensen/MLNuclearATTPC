\documentclass[12pt]{article}
\usepackage{xcolor}
\begin{document}

\section*{Dear Editors,}

we are much indebted for the positive feedback from the reviewer and your willingness to consider our manuscript for publication.
We have tried as faithfully as possible to answer the referee and we hope the revised version meets your standards for publications in NIMA.\newline
Our answers to the referee are included below here. The revised version is included as well.\newline
Yours sincerely,\newline
Daniel Bazin, Morten Hjorth-Jensen, Michelle Kuchera, Robert Solli, and Ryan Strauss

\section*{Answers to the referee}

We are very much indebted for the positive comments on our manuscript and we would like to  thank the reviewer for a careful read of our work. We have tried as faithfully as possible to answer the various comments. We have included them here for the sake of completeness, together with our responses
(highlighted in blue). We hope our responses  aid in improving the quality of the manuscript and thank the reviewer again for her/his willingness for considering our manuscript again.


\begin{itemize}
\item Reviewer comment: Line 29, Simulated data? Line 193 simulated events… from a Ar(p,p) experiment. Are these
simulation data or experimental data? The use of word simulated is confusing. Based on the context,
it appears that only experimental data is used for this work.\\
\textcolor{blue}{Our answer: We apologize for the confusion. Both simulated events and records from the AT-TPC were studied. Simulated events were generated from an event simulator written for the purpose of analysing the Ar(p,p) reaction. We added a sentence in the abstract to clarify the circumstance. }
\item Reviewer comment: Line 124-125, simulated via an integration, which is numerically costly. Rephrase and clarify how
costly quantitatively. Similar clarifications needed in Line 131, Line 145 and Table 2.\\
\textcolor{blue}{Our answer: More quantitative information about the computational cost of traditional analysis methods was added in the paragraphs mentioned by the referee.}
\item Reviewer comment: Figure 1. Add some discussions about the experimental information and physics behind the
observed patterns. For example, which direction is the incoming beam? Beam energy? Vertex at
(0,0)? The direction of the magnetic field?\\

\textcolor{blue}{Our answer: Additional text has been added in the figure caption to better explain the experimental conditions of the displayed data.}
\item Reviewer comment: Line 306, Sample $\hat{x}$, presumably each of these is a 2D image x-y? what are the sample
dimensions?

\textcolor{blue}{Our answer: The text has been modified to clarify that the samples are images encoded as a square matrix with 128 elements along one axis.}
    
\item Reviewer comment: Lines 330 \& 331, define Ground truth labels y and predictions $\hat{y}$, how are they derived from
the raw data (2D projections of the AT-TPC)? Or the latent layer outputs of a pre-trained neural
network? Or else?
\textcolor{blue}{Our answer: We have added more text in each subsection of Section 3: Data Preparation was added to indicate where the ground truth labels were derived from. In fact, we used this data set {\it because} we had rare access to ground truth labels for experimental data.}
\item Reviewer comment: Figures 4, Figure 5, Figure 6, Figure 9, Figure 10, are data science metric results. Needs update or
move some of them to the appendix.
\textcolor{blue}{Our answer: We have followed the reviewer's suggestions and have moved the confusion matrix figures to the appendix, and we added text to clarify their meaning.}
\item Reviewer comment: Figure 7, Figure 8, elucidate the physics and understanding behind the pattern would be helpful.
Similar to the comments to Figure 1 above.
\textcolor{blue}{Our answer: These figures are now the new Figures 4 and 5. Since this is unsupervised clustering, it is normally difficult to  conjecture a clear   physical meaning behind the separation. However, the plots visualize a sampling of proton events that were attributed to a noisy and pure cluster (each cluster), allowing a reader to infer the behavior of the unsupervised learning on these events. The figure caption texts have been updated to clarify this point. }
\end{itemize}

We hope our answers have addressed properly the comments of the reviewer. \newline
Yours sincerely,\newline
The Authors

\end{document}