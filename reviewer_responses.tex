\documentclass[12pt]{article}
\usepackage{xcolor}
\begin{document}

This article presents unsupervised machine learning methods and results for event separation in
the Active-Target Time Projection Chamber (AT-TPC). This work extends a previous related
study [Ref. 6, published in NIMA 2019] from supervised machine learning to unsupervised
learning. Some of the potentially attractive feature of unsupervised learning could include
bypassing the labeling step, forming classes of events independently from the experimenter’s input,
and processing larger amounts (~ TB) of data efficiently. By filtering out un-interesting events, an
unsupervised classification algorithm would also offer the possibility to discover rare events which
may not be expected or are overlooked. These rare events would likely be filtered out using the
traditional methods. The c2 approach used in the traditional analysis performed on the 46Ar data
could be computationally expensive because it involves the simulation of thousands of tracks for
each recorded event.


I am in favor of the publication of the paper in NIMA, subject to clarifications of the terms used
and data method descriptions in the manuscript. Furthermore, the discussions about the results
have mainly been focusing on data method metrics. It would be very helpful to connect or
extrapolate the results to the physics models and understanding or traditional instrumentation
metrics such as ground truth derived from physics models, signal to noise, noise reduction, etc.,
since NIMA is not a data science journal, at least not yet.


Below are some examples.
\begin{itemize}
\item Line 29, Simulated data? Line 193 simulated events… from a Ar(p,p) experiment. Are these
simulation data or experimental data? The use of word simulated is confusing. Based on the context,
it appears that only experimental data is used for this work.\\
\textcolor{blue}{We apologize for the confusion. The events referred to in section 3.1 are simulated, not real. They were generated from an event simulator written for the purpose of analysing the Ar(p,p) reaction. The phrasing of this sentence is modified to clarify the information.}
\item Line 124-125, simulated via an integration, which is numerically costly. Rephrase and clarify how
costly quantitatively. Similar clarifications needed in Line 131, Line 145 and Table 2.\\
\textcolor{blue}{More quantitative information about the computational cost of traditional analysis methods was added in the paragraphs mentioned by the referee.}
\item Figure 1. Add some discussions about the experimental information and physics behind the
observed patterns. For example, which direction is the incoming beam? Beam energy? Vertex at
(0,0)? The direction of the magnetic field?\\

\textcolor{blue}{Additional text has been added in the figure caption to better explain the experimental conditions of the displayed data.}
\item Line 306, Sample $\hat{x}$, presumably each of these is a 2D image x-y? what are the sample
dimensions?

\textcolor{orange}{The text has been modified to clarify that the samples are images encoded as a square matrix with 128 elements along one axis.}
    
\item Lines 330 \& 331, define Ground truth labels y and predictions $\hat{y}$, how are they derived from
the raw data (2D projections of the AT-TPC)? Or the latent layer outputs of a pre-trained neural
network? Or else?
\textcolor{blue}{Additional text in each subsection of Section 3: Data Preparation was added to indicate where the ground truth labels were derived from. In fact, we used this dataset {\it because} we had rare access to ground truth labels for experimental data.}
\item Figures 4, Figure 5, Figure 6, Figure 9, Figure 10, are data science metric results. Needs update or
move some of them to the appendix.
\textcolor{blue}{The confusion matrices have been moved to the appendix, and language was added to clarify their meaning.}
\item Figure 7, Figure 8, elucidate the physics and understanding behind the pattern would be helpful.
Similar to the comments to Figure 1 above.
\textcoloe{blue}{Now Figures 4 and 5.}
\end{itemize}
\end{document}