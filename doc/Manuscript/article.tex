%robert was here sup robert -MIchelle was also here  


%\documentclass[preprint,12pt]{elsarticle}
% uncomment for submission to NIM A
\documentclass[review,number,sort&compress]{elsarticle}
\usepackage{lineno}
\linenumbers


\usepackage{amssymb}
\usepackage{amsmath}
\usepackage{float}
\usepackage{listings}
\usepackage{amssymb}
\usepackage{tabularx}
\usepackage{tikz}
\usepackage{float}
\usepackage{mathtools}
%\usepackage{minted}
\usepackage[colorinlistoftodos]{todonotes}
\usepackage[ruled,vlined]{algorithm2e}
\usepackage{hyperref}
\usepackage{enumerate}
\usepackage[shortlabels]{enumitem}
\usepackage{amssymb}
\usepackage{subfig}
\usepackage{makecell}
\usepackage{array}
\usepackage{multirow}
\usepackage{mathpazo}
%\usepackage{multicol}
%\usepackage{siunits}
\usepackage{siunitx}
\usepackage{graphicx}
\usepackage{longtable}
\usepackage[toc,page]{appendix}
\usepackage{booktabs}
\usepackage{tabularx}
\usepackage{adjustbox}
\usepackage[section]{placeins}
\usepackage{pdflscape}
\usepackage[utf8]{inputenc}
\usepackage[T1]{fontenc}

\usepackage{natbib}
\bibliographystyle{unsrtnat}


\setlength{\marginparwidth}{4cm}
\usepackage{todonotes}
\newcommand{\inner}[2]{\langle #1 | #2 \rangle}
\newcommand{\R}{\mathbb{R}}
\newcommand{\wij}{W_{ij}}
\newcommand{\loss}{\mathcal{L}}

\journal{Nuclear Instruments and Methods
in Physics Research Section A: Accelerators, Spectrometers,
Detectors and Associated Equipment}

\begin{document}

\begin{frontmatter}

%% \title{Title\tnoteref{label1}}
%% \tnotetext[label1]{}
%% \author{Name\corref{cor1}\fnref{label2}}
%% \ead{email address}
%% \ead[url]{home page}
%% \fntext[label2]{}
%% \cortext[cor1]{}
%% \address{Address\fnref{label3}}
%% \fntext[label3]{}

\title{Latent Variable Clustering for Classification of Events in the AT-TPC}

%% use optional labels to link authors explicitly to addresses:
%% \author[label1,label2]{}https://www.overleaf.com/project/5e73f990fcbd310001959b01
%% \address[label1]{}
%% \address[label2]{}

\author{R.~Solli}
\address{Expert Analytics AS, Tordenskiolds gate 6, 0160, Oslo, Norway}
\address{Department of Physics, University of Oslo, POB 1048 Oslo, N-0316 Oslo, Norway}

\author{D.~Bazin}
\address{Department of Physics and Astronomy and Facility for Rare Ion Beams and National Superconducting Cyclotron Facility, Michigan State University, East Lansing, MI 48824, USA}
\author{M.P.~Kuchera}
\address{Department of Physics, Davidson College, Davidson, North Carolina, USA}
\author{R.R.~Strauss}
\address{Department of Mathematics and Computer Science, Davidson College, Davidson, North Carolina, USA}

\author{M.~Hjorth-Jensen}
\address{Department of Physics and Astronomy and Facility for Rare Ion Beams and National Superconducting Cyclotron Facility, Michigan State University, East Lansing, MI 48824, USA}
\address{Department of Physics and Center for Computing in Science Education, University of Oslo, POB 1048 Oslo, N-0316 Oslo, Norway}
\ead{hjensen@frib.msu.edu}
\ead[url]{http://mhjgit.github.io/info/doc/web/}


\begin{abstract}
% Modified by DB (6/8/2020)
This article presents novel applications of machine learning methods to the problem of event separation in an active target detector, the Active-Target Time Projection Chamber (AT-TPC) \cite{Bradt2017}. %Machine learning presents an interesting avenue for researchers operating the AT-TPC, as traditional analysis methods of AT-TPC data are computationally expensive and have to fit all tracks against the event type of interest. The latter presents a considerable challenge when either the space of reactions is not known prior to the analysis, or the event type of interest is not the most abundant. 
The application of unsupervised clustering  algorithms to the analysis of two-dimensional projections of particle tracks from a resonant proton scattering experiment on $^{46}$Ar is introduced. We explore the performance of autoencoder neural networks and a pre-trained VGG16 \cite{Simonyan2014} convolutional neural network.
%on two tasks: a semi-supervised classification task and the unsupervised clustering of particle tracks. 
%On the semi-supervised task, we find that a logistic regression classifier trained on small labelled subsets of the latent space of these models perform very well, both on simulated and real data. On the clustering task, 
We find that a $K$-means algorithm applied to the simulated data in the VGG16 latent space forms almost perfect clusters. Additionally, the VGG16+K-means approach finds high purity clusters of proton events for real experimental data. We also explore the application of neural networks to clustering by implementing a mixture of autoencoders algorithm, with improved results on real data. %
%However, the neural network clustering suffers from stability issues. %necessitating further investigations into this approach. 

%In this work we introduce the application of convolutional autoencoder neural networks to the analysis of two-dimensional projections of particle tracks from a resonant proton scattering experiment on ${}^{46}$Ar. 
%The data we analyze were recorded by an active target time-projection chamber (AT-TPC). Machine learning presents an interesting avenue for researchers operating an AT-TPC, as traditional analysis methods of AT-TPC data are both computationally expensive and fit all particle tracks against the event type of interest. The latter presents a considerable challenge when the space of reactions is not known prior to the analysis. 

%We explore the performance of the autoencoder neural networks and a pre-trained VGG16 \cite{Simonyan2014} convolutional neural network on two tasks: a semi-supervised classification task and the unsupervised clustering of particle tracks. On the semi-supervised task, we find that a logistic regression classifier trained on small labelled subsets of the latent space of these models perform very well. On simulated data these classifiers achieve an $f1$ score \cite{Chinchor1992} of $f1>0.95$. The VGG16 latent classifier achieves this result with as few as $N=100$ samples, as does the convolutional autoencoder when trained on the VGG16 representations of the particle tracks. On real data, pre-processed with noise filtering, the same models achieve an $f1>0.7$. For unfiltered real data the models achieve an $f1$ value larger than $0.6$. Both of the previous results were found with the classifiers trained on $N=100$ samples. Furthermore, we found that the autoencoder model reduces the variability in the identification of proton events by $64\%$ from the benchmark logistic regression classifier trained on the VGG16 latent space on real experimental data. 

%On the clustering task, we found that a $K$-means algorithm applied to the simulated data in the VGG16 latent space forms almost perfect clusters, with an adjusted rand index \cite{Hubert1985} ($ARI$) $ > 0.8$.  Additionally, the VGG16+K-means approach finds high purity clusters of proton events for real experimental data. We also explore the application of neural networks to clustering by implementing a mixture of autoencoders algorithm. With this model we improved clustering performance on the real experimental data from an $ARI = 0.17$ to an $ARI = 0.40$. However, the neural network clustering suffers from stability issues necessitating further investigations into this approach. 

\end{abstract}

%%Graphical abstract
\begin{graphicalabstract}
%\includegraphics{grabs}
\end{graphicalabstract}

%%Research highlights
\begin{highlights}
\item Research highlight 1
\item Research highlight 2
\end{highlights}

\begin{keyword}
%% keywords here, in the form: keyword \sep keyword

%% PACS codes here, in the form: \PACS code \sep code

%% MSC codes here, in the form: \MSC code \sep code
%% or \MSC[2008] code \sep code (2000 is the default)

\end{keyword}

\end{frontmatter}

\section{Introduction}\label{sec:intro}

%
%\begin{itemize}
%    \item Overall goals for the AT-TPC
%    \item Nuclear physics motivation for clustering-based (unsupervised) %methods
%   \item Motivation from ML approach
%   
%\end{itemize}

The AT-TPC \cite{Bradt2017} is a novel type of detector designed specifically for nuclear physics experiments where the energies of the recoiling particles are very low compared to the energy required to escape target material. The luminosity of nuclear physics experiments performed with fixed targets is directly proportional to the amount of material encountered by the beam. On the other hand, for several classes of experiments the detection of recoil particles is paramount, therefore limiting the target thickness. In addition, the properties of the recoil particles are modified while traversing the target material, affecting the resolutions that can be achieved. This necessary balance between luminosity and resolution is particularly difficult when performing experiments with rare isotope beams, because of the low intensities available. 

The concept of active targets aim at mitigating this compromise, by turning the target itself into a detector \cite{BECEIRONOVO2015}. Most active target detectors such as the AT-TPC are composed of a time projection chamber (TPC) where the detector gas is at the same time the target material. Recoil particles that originate from a nuclear reaction between a beam nucleus and a gas nucleus can be tracked from the vertex of the reaction to their final position inside the active volume of the target. Their properties can therefore be measured without any loss of resolution regardless of the amount of material traversed by the beam. At the same time, the detection efficiency is dramatically increased by the very large solid angle covered in this geometry. A direct consequence of this concept is the inclusiveness of the experimental data recorded by this type of detector: any nuclear reaction happening within the target is recorded. Although this sounds like an advantage from the scientific point of view, it poses great challenges during the analysis phase, that are reminiscent of bubble chamber times and on par with event classification challenges in particle physics today, see for example the recent review of Mehta {\em et al.} \cite{mehta2019}. More often than not, the reaction channel of interest has one of the lowest cross sections, therefore the physicist is faced with the task of sorting out the corresponding events from the ``background" of other reaction channels. 

% motivating ML work:
Because TPCs produce 3-dimensional images of charged particle tracks, the event identification task is often akin to a visual inspection (akin to analysis in the bubble chamber era), which is not practical nowadays because of the large quantities of data. Machine learning techniques then appear as a promising prospect, in particular in the image recognition domain where much progress has been made recently \cite{mehta2019}. In addition, some ML algorithms offer new possibilities such as the potential discovery of unforeseen phenomena that would have been missed by more traditional analysis methods. Prior work has demonstrated the ability to apply supervised classification machine learning methods to AT-TPC data when a labeled training set is available, whether through hand-labeled data or labeled simulated data (a {\em transfer learning} application) \cite{Kuchera2019}. In some experiments, a labeled dataset is unavailable. This could be  due to the inability to hand-label events, or the case where one does not know a priori the types and behaviors of the reactions present in the detector in order to generate a labeled, simulated dataset. In the latter case, there must also exist a validation dataset of real data, still requiring the ability to label a subset of the real data. The unsupervised separation of event types, or {\em clustering}, based on a set of ML algorithms is hereby examined, using experimental data recorded by the AT-TPC during its commissioning experiment from a radioactive $^{46}$Ar beam reacting on an isobutane target composed of proton and carbon nuclei. 

\section{Experimental Details} 
The goal of the experiment was to measure the evolution of the elastic and inelastic cross sections between $^{46}$Ar and protons as a function of energy (the {\em excitation function}), and observe resonances in the composite system $^{47}$K that correspond to analog states in the nucleus $^{47}$Ar. Spectroscopic information can then be obtained from the shape and amplitude of the observed resonances \cite{Bradt2018}. 
The experiment was performed at the National Superconducting Cyclotron Facility (NSCL) where a $^{46}$Ar beam was produced via fragmentation of a $^{48}$Ca beam on a $^9$Be target at about 140 MeV/u. The $^{46}$Ar isotopes were then filtered, thermalized, and finally re-accelerated to 4.6 MeV/u by a linear accelerator. This scheme was used to produce a low-emittance beam, which is necessary to guarantee a good energy resolution in the excitation function. Because the  $^{46}$Ar beam particles lose energy as they traverse the target gas volume, the position of the reaction vertex along the beam axis is directly related to the energy at which the reaction occurs. This allows the AT-TPC to measure the excitation function over a wide range of energies from a single beam energy.

The detector was placed inside the bore of a MRI solenoid energized to $\sim 2$ Tesla. This axial magnetic field served the purpose of bending the trajectories of the recoil particles in order to i) increase their length and ii) provide a measurement of their bending radius, directly related to their magnetic rigidity. Because the recoil particles travel in gas, they slow down and eventually stop, therefore their trajectories are described by 3-dimensional spirals (see \cite{Bradt2017}). One of the difficulties encountered in the analysis is that the shape of these spirals does not have an analytical form because it follows the energy-loss profile of the particles. It therefore needs to be simulated via an integration, which is numerically costly. Other difficulties are related to several experimental effects that deteriorate the quality of the data, namely saturation and cross-talk effects, as well as random noise. 

The method used in \cite{Bradt2018} to analyze the data followed a 3-phase sequence: cleaning, filtering and fitting. Traditional methods were used to perform each of these tasks, and ultimately extract the scientific information, but there were severe limitations and high computational costs that become prohibitive in data sets larger by an order of magnitude. 
% Begin added by DB (06/10/2020)
The cleaning was performed using a combination of linear and circular Hough transforms on a 2-dimensional projection of the tracks \cite{Bradt2017}. The following filtering and fitting phases were performed simultaneously, by defining the cost function to the fitting algorithm as a sum of 3 $\chi^2$ components based on i) the position of the track in space, ii) the energy deposited on each pixel of the sensor plane, and iii) the location of the vertex of the reaction. While various fitting algorithms were tested, the most accurate was a Monte-Carlo algorithm that explored the 6-dimensional phase space of the particle's kinematics parameters, reducing it progressively at each iteration step until the desired accuracy was reached \cite{Bradt2017}. Although this algorithm ended up being the most accurate, it is extremely costly computationally because of the very large number of simulated tracks needed for each event. The filtering was performed by setting limits to the $\chi^2$ distributions, below which the events were assigned as proton scattering. This is a very inefficient method because it requires performing the fitting for all events, including those that are not of interest. Pioneering work on event identification using ML methods, namely the use of a pre-trained CNN, later showed that the filtering phase would better be performed using this type of technique \cite{Kuchera2019}. 
% End added by DB (06/10/2020)
In addition, \cite{Kuchera2019} demonstrated that the purity and statistics of the data are improved with the use of ML.


\section{Data Preparation}

In this section we give a brief overview of the data, for a more in-depth consideration we refer the reader to \cite{Mittig2015}, \cite{Suzuki2012} and  \cite{Bradt2017a}. 

The AT-TPC data we studied for this work was recorded as charge time-series for each of the  the $\sim10^4$ detector pads.
In this representation, an event is a record of 512 time-buckets for each of the detector pads. In our analysis, we represent each event as a downsampled 2D projection.
We chose to represent the data in 2D to facilitate the use of sophisticated image-recognition machine learning models.
The 2D representation of events was computed by viewing the event as a point cloud in the charge time-series and summing over the time-axis.
Additionally, we log-transform and normalize the data to the interval $[0, 1]$ for each point. The final representation for each event is a $128 \times 128$ matrix with each element ranging from $0$ to $1$.

One of the significant considerations for the analysis of AT-TPC data is to inject machine learning methods for track identification at the best point in the analysis pipeline.
Using raw data is advantageous as it provides an unbiased view of the event, but the data volume and noise levels might be prohibitive for analysis.
%On the opposite end of the pipeline, 
Therefore, we will incrementally add bias to the analysis by applying the algorithm further down the analysis pipeline.. With the benefit being that more pre-processing improves the signal-to-noise ratio, possibly improving model performance.
To explore this trade-off between model performance and pre-processing bias, we performed analysis on simulated, cleaned and raw events.

\begin{table}
\centering
\caption{Event class percentages for each of the datasets used in this analysis. The decrease in the "other" class of events from full to filtered data owes to the thresholding of events. If an event contains fewer than $20$ data-points it is discarded }\label{tab:class_distr}
\begin{tabular}{lccc}
\toprule
{} & Simulated & Full & Filtered \\
\midrule
$\%$ Proton & $50$ & $25.3$ & $27.6$ \\ 
$\%$ Carbon & $50$ & $12.0$ & $11.9$ \\
$\%$ Other & $0$ & $62.8$ & $60.1$ \\
\end{tabular}
\end{table}

\subsection{Simulated \texorpdfstring{${}^{46}$Ar}{46Ar}  events}\label{sec:data_sim}

AT-TPC tracks were simulated with the \lstinline{pytpc} package developed by \citet{Bradt2017a}. Using the same parameters {\color{blue} what parameters are you referring to? -MPK} as for the ${}^{46}$Ar$(p, p)$ experiment, a small set of $N=4000$ events per class was generated, as well as a larger set of $N=40000$ events per class. 

To emulate the real-data case, we set a subset of the simulated data to be labelled and treat the rest as unlabelled data. We choose this partition to be $15\%$ of each class. We denote this subset and its associated labels as $\gamma_L=(\boldsymbol{X}_L, \boldsymbol{y}_L)$, while the entire dataset which is identified as $\boldsymbol{X}_F$. Note that $\boldsymbol{X}_L \subset \boldsymbol{X}_F$.

Two simulated events are displayed in figure \ref{fig:sim_samples}. The top row illustrates a proton-event, and the bottom a carbon-event. 



\subsection{Raw \texorpdfstring{${}^{46}$Ar}{46Ar}  events}\label{sec:data_real}

The events analyzed in this section were retrieved from the ${}^{46}$Ar resonant proton scattering experiment recorded with the AT-TPC. 
While we denote these events as raw, it is important to note that what we mean is a raw 2D representation of peak-only events.

%The sensor plane in the AT-TPC is very sensitive, as such there is substantial noise in the ${}^{46}$Ar data. The noise can be attributed to structural noise from electronics cross-talk, and possible interactions with cosmic background radiation, as well as other sources of charged particles. Part of the challenge for this data then comes from understanding of the physics of the major contributing factors to this noise. \todo{I guess this paragraph can go. It reads very thesis-y, but I'm not sure what to do here to sensibly frame the raw data}

We display two different events from the ${}^{46}$Ar experiment in figure \ref{fig:samples}. The top row illustrates a carbon event with a large fraction of noise, while the bottom row shows a proton event nearly devoid of noise.

\begin{figure}[ht]
\centering
\includegraphics[width=\textwidth]{custom_work/examples_raw.pdf}
\caption[Displaying unfiltered events in 2D and 3D]{Two- and three-dimensional representations of two events from the ${}^{46}$Ar experiment. Each row is one event in two projections, where the color intensity of each point indicates higher charge values recorded by the detector.}\label{fig:samples}
\end{figure}

\subsection{Filtered \texorpdfstring{${}^{46}$Ar}{46Ar} events}\label{sec:filtered}


As we saw in the previous section, the detector picks up a significant amount of noise. We split the noise broadly in two categories,  one being random-uncorrelated noise and the second is structured noise. The former can be quite trivially removed with a nearest-neighbour algorithm that checks if a point in the event is close to any other. To remove the correlated noise, researchers at the NSCL developed an algorithm based on the Hough transform \cite{Newman1972}. This transformation is a common technique in computer vision, used to identify common geometric shapes like lines and circles, and has been used extensively in high-energy particle physics since the bubble-chamber era \cite{Hough:1959}.  %Essentially, the algorithm draws many lines (of whatever desired geometry) through each data-point and checks whether these lines intersect with points in the dataset. Locations in parameter space that generate many intersections then become bright spots, allowing us to filter away points that are not close to these points. These algorithms remove a large amount of the unstructured noise and are computationally rather cheap. \todo{Also pretty thesis-y language. Do not like}

We illustrate two filtered events in figure \ref{fig:samples_filtered}. These are the same events as shown in figure \ref{fig:samples}, but with the Hough and nearest neighbours filtering applied. 

\begin{figure}[ht]
\centering
\includegraphics[width=0.9\textwidth, height=9cm]{custom_work/examples_filtered.pdf}
\caption[Displaying filtered events in 2D and 3D]{Two- and three-dimensional representations of two events from the ${}^{46}$Ar experiment. Each row is one event in two projections, where the lightness of each point indicates higher charge values. These events have been filtered with a nearest neighbors algorithm and an algorithm based on the Hough transform, described in section \ref{sec:filtered}}\label{fig:samples_filtered}
\end{figure}


\section{Methods}\label{sec:methods}

\subsection{Classifying  events} 
\todo{Brute force serving as motivation for why we do unsupervised learning.}
%The filtering task relied on a set of $\chi^2$ distributions based on different criteria, 
The traditional event selection process involved a Monte-Carlo $\chi^2$ minimization method, however there was no well-defined method to quantify the effectiveness of the event selection from the thresholds applied to these distributions.
{\color{blue} I think we discussed this method earlier, but I need to verify. otherwise needs more explanation -MPK}
In addition, the selection task produced a binary result only, ``good" or``bad", relative to the event of interest, while the data clearly contained more classes of events%, if only because of the presence of carbon in the target (from the isobutane gas).
In a broader perspective, an unsupervised classification algorithm would offer the possibility to ``discover" rare events not anticipated (or overlooked) by the physicist. These events would likely be filtered out using the traditional methods. From a practical point of view, compared to supervised learning, it also avoids the necessary labeling task of the learning set events, which is error prone and time consuming.

\subsection{Why machine learning}

\begin{itemize}
    \item Traditional MC methods fall short in two principal ways:
    \begin{itemize}
        \item The computational cost per event is too large given the size of the data-sets
        \item The broken tracks, noisy environment creates bad fit statistics for otherwise useful events. End result is $f1 \sim 0.7 $
        \item define a ``bad" fit as a non-proton event
    \end{itemize}
\end{itemize}

The $\chi^2$ approach used in the traditional analysis performed on the $^{46}$Ar data is extremely computationally expensive because it involves the simulation of thousands of tracks for each recorded event, which are performed each iteration of the Monte-Carlo fitting sequence. Even though the reaction of interest in the $^{46}$Ar experiment had the largest cross section (elastic scattering), the sheer number of events to analyse (in the hundreds of thousands) necessitated the use of parallel processing on a high performance computing cluster {\color{blue} is there a metric we can give here? -MPK}. In the case of an experiment where the reaction of interest would represent less than a few percent of the total cross section, this procedure would become highly inefficient and prohibitive. 
%Most of the CPU time would be spent on simulating events that are not of interest.
The computationally expensive fitting procedure would be applied to every event, instead of the few percent of the events that are of interest for the analysis.
An unsupervised ML algorithm able to separate the data without a priori knowledge of the different types of events increases the efficiency of the analysis tremendously, and allows the downstream analysis to concentrate on the fitting efforts only on events of interest. In addition, the clustering allows for more exploration of the data, potentially enabling new discovery of unexpected reaction types.

%\subsection{Data}
%{\color{green}What is the intent of this section? Display some data examples to illustrate the points above?}


%{\color{orange} Yes, but also this section is a necessary bridge for reproducibility. Maybe it's better as an appendix?}

\begin{table}[hbtp]
\centering
\caption{Descriptions of number of events in the data.}\label{tab:datasets}
\begin{tabular}{lccc}
\toprule
{} & Simulated & Full & Filtered \\
\midrule
Total &  $8000$ & $51891$ & $49169$ \\
labelled & $2400$ & $1774$ &  $1582$ \\ 
\bottomrule
\end{tabular}
\end{table}

%\begin{itemize}
%     \item Reiterate the aim of unsupervised  clustering of events.
%     \item Challenges from a machine learning perspective. 
%     \begin{itemize}
%         \item Supervised vs. Unsupervised learning
%         \item Define Classification and Clustering 
%         \item Define latent space (refer)
%         \item Traditional unsupervised - distances in high dimensional spaces etc. 
%     \end{itemize}
%     \item Digression here to explain some ML concepts? Link back to Conv part of Michelles paper? 
%     \item Opportunities presented by: 
%     \begin{itemize}
%         \item Transfer learning (train on VGG use on AT-TPC)
%         \item Compression is understanding (leverage autoencoders) 
%     \end{itemize}
%     \item Building on Kuchera et. al we investigate the VGG16 networks application. 
%     \item Novel contribution by applying clustering autoencoder networks. 
% \end{itemize}

\subsection{Pre-trained neural networks}

% To affect the clustering of AT-TPC events, we first considered the results from \citet{Kuchera2019}. In their work, the authors consider the segmentation of reactions in the $^{46}$Ar experiment as a classification problem. Given ground-truth labels, they apply convolutional neural networks (CNN) to the problem. Using a CNN was in large part inspired by results from high-energy physics and, in particular, the work by \citet{Aurisano2016}. 

% Using a CNN algorithm comes with the caveat that the data should be image-like. This caveat is especially stringent when applying pre-trained models to new data. For the $^{46}$Ar data we project the 3D point clouds in each event to a 2D matrix representation of the X-Y plane, summing over the Z-axis. Applying a pre-trained model is a common technique in computer science, and \citet{Kuchera2019} showed that a pre-trained CNN model outperforms bespoke models for the classification of $^{46}$Ar data. We build on the classification result in this work by using a pre-trained CNN model as a spring-board for our clustering models.

% In the Machine Learning community, it is not uncommon to publish packaged models with fitted parameters from image recognition contests. These models are trained on datasets with millions of images and classify between hundreds of distinct classes; one such is the imagenet dataset. In their work, \citet{Kuchera2019} used the VGG16 architecture trained on imagenet to classify ${}^{46}$Ar data. In our work we also found the VGG16 architecture to capture the strongest event-separating representation for clustering purposes. 

% The VGG16 network is one of six analogous networks proposed by \citet{Simonyan2014}. The network architectures are fairly straightforward; for VGG16, there are sixteen layers in the network. The first thirteen of which are convolutional layers with exclusively $3 \times 3$ kernels.  The full architecture is detailed in appendix \ref{app:vgg}.

Training high-performing neural networks from scratch often requires enormous datasets and computation time. However, it has been found that models which are trained at large scale will learn general features that are applicable to a variety of tasks. For example, large neural networks which are trained on the ImageNet dataset \cite{Russakovsky2015} --- a diverse image classification task --- learn how to identify lines, edges, and other common shapes which are useful for numerous problems. Thus, it is common practice to initialize the convolutional layers of a network with the pre-trained weights learned from ImageNet (or some other large dataset). The training process then only has to fine-tune the network for the specific task. Since we are building on prior knowledge in this case, learning becomes far more efficient and better performance can often be achieved. \citet{Kuchera2019} used machine learning methods to classify the products of $^{46}$Ar reactions in the AT-TPC, and they found that a CNN initialized with ImageNet weights resulted in the most successful classification.

\subsection{Clustering on latent spaces}

% In contrast with the classification work of \citet{Kuchera2019}, we do not append a classifying layer to the VGG16 network given that we do not have access to the ground truth labels at training time in the clustering regime. We instead consider the output from the convolutional blocks of the VGG16 network as a representation of the events on which we will apply a clustering algorithm. 

% The clustering algorithm we found to give the strongest result is the K-means algorithm. This algorithm is contingent on having a distance measurement between two datapoints. We found that the standard choice of a euclidian norm on the vector differences provided the best outcomes.

In contrast with the classification work of \citet{Kuchera2019}, we do not assume access to ground truth labels and are trying to solve a fundamentally different learning problem. Thus, rather than fine-tuning a pre-trained network under the supervised learning regime, we extract the output of the pre-trained network's last convolutional layer as a latent representation of the events. We then cluster events based on this representation using the K-means++ algorithm with the standard choice of Euclidean distance.

Building on the work by \citet{Kuchera2019}, we also use the VGG16 pre-trained network as a baseline for the clustering performance. We begin by considering a classical K-means approach to clustering. However, the output from the VGG16 network is very high dimensional which traditionally conflicts with good K-means results. With output vectors in $\R^{8192}$ the ratio between $L_p$ distances goes to one with increasing dimensionality, as shown by \citet{Aggarwal}. However, one of the central assumptions in the authors' finding is that the elements are uniformly distributed in the latent space. They note that it is then possible that if the class information lies in some sub-space of the latent data, $L_p$-norms may provide useful clusterings even in high dimensional spaces. To investigate this, we perform a $L_2$-norm based K-means clustering using the full VGG16 representation and the $10^2$ first principal components only. 

%{\color{blue} Are there supposed to be results here? -MPK}
%\subsection{K-means}

%We begin by investigating the K-means clustering algorithm on the VGG16 latent space. 
The VGG16 model is pre-trained on the imagenet dataset, and subsequently applied to the AT-TPC events without the classification layers. This creates a set of vectors $\boldsymbol{x} \in \R^{8192}$ as our new representation of the events. To cluster we use \lstinline{scikit-learn} implementation of the K-means algorithm, with default parameters \cite{Pedregosa2011}.


\subsection{Deep clustering: Mixture of autoencoders}\label{sec:mixae}

As an alternative to relying on a pre-trained model, we also consider bespoke end-to-end clustering models, which are specifically trained on the AT-TPC data. {\color{cyan} [Ryan: I think everything after this point in this paragraph should either be removed or go in a discussion at the end. Because it is entirely about DCEC, which isn't one of the two main methods we are presenting and more immediately isn't MIXAE, which is what this section is about. Or, this section just needs to be presented as something slightly more general, but right now, ``mixture of autoencoders'' is in the title.]} 

The MIXAE algorithm clusters data by pairing a set of encoder-decoder neural networks (autoencoder) with an auxiliary assignment network: each autoencoder constructs a representation of a given event while the assignment network assigns a set of cluster probabilities to the event. {\color[orange]{clear sentences about what the model does}} As noted by \citet{Zhang}, the MIXAE model assumes that the clusters are roughly similar in volume in the data and that one has some reasonable estimate on the number of clusters present {\color{cyan} [Ryan: Is this true of our data? I think if these assumptions are going to be mentioned, then we need to say that our data meets the criteria.}. The architecture is portrayed in figure \ref{fig:mixae}, tapered boxes denote a direction of compression in the network components. 

As the MIXAE model does not have access to ground truth labels the training consists of applying simple heuristics that constitute a necessary, but not sufficient, fulfilment for a high-quality MIXAE model given the assumptions made of the model. The heuristics can be stated as: 
\begin{enumerate}
	\item  Given a clustering assignment to a particular autoencoders' representation that autoencoders' reconstruction should be similar to the input event. 
	\item Within a batch of events presented to the model, there should be a spread in the clustering assignment between all clusters. 
	\item Each clustering prediction should be as close to a strong prediction as possible, i.e. assigning high probabilities is preferable to weak assignments.
\end{enumerate}

\noindent For a more formal consideration of these heuristics see \citet{Zhang}.

{\color{cyan} [Ryan: When I reach the end of this section, I'm still not quite sure its obvious what MIXAE is doing. I don't necessarily know how to reword things yet, but I think a more intuitive initial explanation of what it does could be helpful.}

\begin{figure}[tb]
	\centering
	\includegraphics[width=.8\textwidth]{plots/mixae.pdf}
	\caption[Mixture of autoencoders schematic]{Schematic of a MIXAE model. A sample $\hat{\boldsymbol{x}}$ is compressed to set of lower-dimensional representations $\{\boldsymbol{z}^{(i)}\}$ by $N$ autoencoders. These samples are concatenated and passed through an auxiliary assignment network that predicts a confidence of cluster belonging for each autoencoder. For further details see the text. Figure adapted from \citet{Zhang}}
	\label{fig:mixae}
\end{figure}
{\color{blue} This section ends quite abruptly -MPK}


\section{Results and Discussions}\label{sec:results}

The principal challenge in the AT-TPC experiments that we are trying to solve is the reliance on labelled samples in the analysis as future experiments may not have as visually dissimilar reaction products  as we observe in the ${}^{46}$Ar experiment.  The  ${}^{46}$Ar experiment does, however, provide a useful example where we can then explore unsupervised techniques. 

We begin by exploring the results of applying a K-means approach on the latent space of a pre-trained network. Subsequently, we investigate the performance of the MIXAE algorithm outlined in section \ref{sec:mixae}.

\subsection{K-means clustering on the VGG16 latent space}

The results of the clustering runs are included in table \ref{tab:clstr_vgg}. We observe that we are able to attain near-perfect clustering on simulated data and that there is a sharp decline in performance as we add noise by moving to the filtered and raw datasets. 


\begin{table}[H]
\centering 
\caption[K-means on pre-trained model]{K-means clustering results on AT-TPC event data. We observe that the performance predictably decreases with the amount of noise in the data.}\label{tab:clstr_vgg}
\begin{tabular}{lll}
\toprule
{} & Accuracy &   ARI \\
\midrule
Simulated &     0.70 &  0.89 \\
Filtered  &     0.75 &  0.39 \\
Raw       &     0.59 &  0.17 \\
\bottomrule
\end{tabular}

\end{table}

In addition to the performance measures reported in table \ref{tab:clstr_vgg}, it is interesting to observe which samples are being wrongly assigned. We achieve this by tabulating the assignments of samples relative to their ground truth labels. From these tables, we can infer which classes are more or less entangled with others. We tabulate the results for each dataset is in figure \ref{fig:clster_confmat}. We observe that the proton class is consistently assigned in a pure cluster. Purity is inferred by how much spread there is in the column between the ground truth labels. A high-quality cluster will, in addition to being pure, also capture most entries the class represented by the cluster. For example, consider the row corresponding to the proton class in figure \ref{fig:clster_confmati_filt}. The column corresponding to the largest entry in the proton row has zero other predicted classes in it. From this, we conclude that the proton cluster is a high quality, high purity cluster. 

This high quality cluster also appears in the clustering of raw data. From figure \ref{fig:clster_confmati_raw} we observe that there is a high purity proton cluster. In contrast to the filtered data we observe that the deterioration in performance can largely be ascribed to the algorithm creating a proton + other cluster and a carbon + other cluster.

\begin{figure}
\centering
	\includegraphics[width=\textwidth]{custom_work/Simulatedvgg_conf_mat.pdf} 
	\caption[Pre-trained network - confusion matrices]{Confusion matrix for the K-means clustering of simulated AT-TPC events. The true labels indicate samples belonging to the p (proton), or the carbon (C) class}\label{fig:clster_confmat_sim}
\end{figure}

\begin{figure}
\centering
	\includegraphics[width=\textwidth]{custom_work/Filteredvgg_conf_mat.pdf}
	\caption[Pre-trained network - confusion matrices]{Confusion matrix for the K-means clustering of filtered AT-TPC events. The true labels indicate samples belonging to the p (proton), carbon (C), or other classes. }\label{fig:clster_confmati_filt}
\end{figure}

\begin{figure}
	\includegraphics[width=\textwidth]{custom_work/Rawvgg_conf_mat.pdf}
\caption[Pre-trained network - confusion matrices]{Confusion matrix for the K-means clustering of raw AT-TPC events. The true labels indicate samples belonging to the p (proton), carbon (C), or other classes. }\label{fig:clster_confmati_raw}
\end{figure}

We repeat this analysis using a Principle Component Analysis (PCA) dimensionality reduction \footnote{
PCA is a common techinque to find the significant variations in data by projecting the data along a subset of its covariance matrix eigenvectors \cite{Marsland2009}
} on the latent space of the VGG16 model. This is done to estimate to what degree the class separating information is encoded in the entirety of the latent space, or in some select regions. The results from the PCA analysis were virtually identical to the results sans the PCA, and so we omit them for brevity. 

Furthermore, we wish to characterize further the clusters presented in the matrix in figures \ref{fig:clster_confmati_filt} and \ref{fig:clster_confmati_raw}. To achieve this, we sample from the proton samples belonging to different clusters for the filtered and full data.

\begin{figure}
\includegraphics[width=\textwidth]{custom_work/Filteredexamples.pdf}
\caption[Filtered proton samples by cluster belonging]{Illustrating a sample of proton events from different K-means clusters from the filtered dataset. Each row belongs to a single cluster corresponding to the filtered confusion matrix in figure \ref{fig:clster_confmati_filt}}\label{fig:filtered_vgg_clster_repr}
\end{figure} 

\begin{figure}
\includegraphics[width=\textwidth]{custom_work/Rawexamples.pdf}
\caption[Full proton samples by cluster belonging]{Illustrating a sample of proton events from different K-means clusters from the raw dataset. Each row belongs to a single cluster corresponding to the full confusion matrix in figure \ref{fig:clster_confmati_raw}}\label{fig:full_vgg_clster_repr}
\end{figure} 


\subsection{MIXAE clustering results}

In the previous section we demonstrated a rather naive clustering technique for AT-TPC event classification. To build on this result we will in this section explore the application of the mixture of autoencoders (MIXAE) algorithm introduced in section \ref{sec:mixae}.
Other algorithms for deep clustering based on autoencoder architectures were also implemented. These did not succeed in the clustering task and were omitted for brevity\footnote{The algoritms we implemented were the Deep Embedded Convolutional Clustering \cite{Guo2017} and its predecessor Deep Embedded Clustering \cite{Xie2016}}.

{\color{orange}[move to appendix]}
In the MIXAE algorithm the hyper-parameters to adjust are all the ordinary parameters that we introduced in table \ref{tab:convae_hyperparams}. In addition to those parameters we have the weighting of the loss terms: $\theta$, $\alpha$ and $\gamma$. These weighting parameters are attached to the reconstruction loss, sample entropy and batch-wise entropy respectively \cite{Zhang}. 
We focused on the tuning of the clustering hyper-parameters, and heuristically set the autoencoder hyper-parameters to a shallow $3\times3$ convolutional network.  

The parameters chosen for the autoencoders are listed in full in table \ref{tab:mixe_ae_hyperparams}.

\begin{table}[H]
\centering 
\caption[MIXAE clustering performance]{MIXAE clustering performance on the ${}^{46} Ar$ experimental data. In contrast with the VGG-16 + K-means approach we observe  significant variations in performance.}\label{tab:clstr_vgg}
\begin{tabular}{lllll}
\toprule
{} & \multicolumn{2}{c}{Accuracy} &   \multicolumn{2}{c}{ARI} \\
\midrule
{} & Top 1 &  $\mu \pm \sigma$ init $= 1$ & Top 1 & $\mu \pm \sigma$ \\
Simulated &  0.96 &  $0.74 \pm 0.16 $  & 0.84 &  $0.33 \pm 0.32 $\\
Filtered  & 0.75 &  $0.71 \pm 0.04 $ & 0.52 &  $0.38 \pm 0.14 $\\
Raw & 0.71 &  $0.61 \pm 0.07 $ &  0.32 &  $0.09 \pm 0.10 $   \\
\bottomrule
\end{tabular}
\end{table}

\subsection{Simulated AT-TPC data}


Finally, for these parameters we re-ran the algorithm $N=10$ times to investigate the stability of the algorithm. The results are reported in figure \ref{fig:mixae_sim}. We observe that while the algorithm can achieve very strong performance, with an adjusted rand index (ARI) $ > 0.8$, it fluctuates strongly with repeated experiments. The $ARI$ is a clustering performance measure similar to accuracy, but adjusted for chance. We elaborate on the $ARI$ score in section \ref{sec:unsupervised_perf}. 

We repeat the optimization steps in the previous paragraph for the filtered AT-TPC data. The exception being that we allow the algorithm to train on the labelled samples. Beginning with a wide grid equal to the grid used for the simulated data we searched over all parameter configurations to find promising values. We then performed a local search around these values to pin-point the hyperparameter configuration.

\noindent The results of the runs are included in figure \ref{fig:mixae_clean}. We observe that the highest performing models reach an $ARI > 0.5$, which is higher than the performance achieved by the K-means algorithm applied to the VGG16 latent space. 

We repeated the same procedure described in the two previous paragraphs to the raw dataset. Each configuration was re-run a total of $N=10$ times to capture fluctuations in the performance before a final selection is made on the hyperparameters. In contrast to the previous two results the MIXAE algorith very rarely converges to a high performig configuration. In fact, most runs are indistinguishable from random chance clusterings.

\subsection{Comparing performance}

It is also interesting to compare and contrast the clustering results from the MIXAE model with those of the VGG16$+$K-means outside the fairly abstract accuracies and rand scores. It is especially interesting to compare the cluster assignments, as they can inform further research efforts in the clustering of AT-TPC events. We illustrate the clustering with confusion matrices that are shown in figure \ref{fig:mixae_confmat}. From these matrices, we observe that the MIXAE applied to the clean data correctly clusters the noise events. Additionally, it identifies two proton clusters. We observe that these proton clusters are both less pure than the VGG16+K-means clusters and that there does not seem to be a visually meaningful difference between these clusters. The latter is inferred from figure \ref{fig:filtered_mixae_clster_repr}. 

Applied to the real data the MIXAE correctly separates the proton class, however it is unable to separate the carbon events from the amorphous noise events or from the proton cluster. 

\begin{figure}[H]
\centering
	\includegraphics[width=\textwidth]{custom_work/Filtered_mixae_conf_mat.pdf}
\end{figure}

\begin{figure}[H]
\centering
	\includegraphics[width=\textwidth]{custom_work/Raw_mixae_conf_mat.pdf}
\caption[MIXAE - confusion matrices]{Confusion matrices for the MIXAE clustering algorithm on filtered and full AT-TPC events. The true labels indicate samples belonging to the p (proton), carbon (C), or other classes. }\label{fig:mixae_confmat}
\end{figure}

Lastly we wish to further investigate if there are systematic differences between proton events that were placed in different clusters for the clean and full data. From figure \ref{fig:mixae_confmat} we see that the MIXAE algorithm creates two proton clusters for the filtered data, and places about fifty per-cent of the proton events in a cluster with the amorphous "other" events.

\subsection{Alternative approaches}

In addition to the results presented in this section, we performed clustering with a number of different algorithms included in the \lstinline{scikit-learn} package. None of them provided any notable differences from the K-means results or were significantly worse. Notably, the DBSCAN algorithm \cite{Ester96adensity-based}\cite{Bergstra2012} failed to provide any useful clustering results. We find this important as one of the significant drawbacks of K-means, and the deep clustering algorithm presented in section \ref{sec:mixae}, is that they are all dependent on pre-determining the number of clusters. This is not the case for DBSCAN. 

{\color{orange}[binding sentence]}
We considered the deep convolutional embedded clustering (DCEC) by \citet{Guo2017} as well as the mixture of autoencoders (MIXAE) method introduced by \citet{Zhang}. While we were able to reproduce the authors' results on their data, the DCEC algorithm proved unable to cluster AT-TPC in our implementation. However, this provides valuable insight as the seeds of the clusters are constructed by a K-means algorithm. This insight contrasts with our positive results from applying a pre-trained model with k-means and highlights potentially significant differences in models trained on a supervised or unsupervised objective for clustering tasks in nuclear physics. 


\section{Conclusions and Perspectives}\label{sec{conclusion}}

From our analysis it is clear that we are a ways off from a fully automated unsupervised method for track classification. {\color{orange}[flip this to positive ]} It is also clear that there is significant potential in both avenues explored in this work. 
In particular, the surprising ability of the K-means algorithm in picking out clear proton clusters from the VGG16 latent space lends itself well to an exploratory phase of analysis.
Another surprising facet of the K-means clustering is its consistent performance. As shown in table \ref{tab:clstr_vgg}, the variance is zero for the performance metrics. This result indicates that the clusters are very clearly defined in the VGG16 latent space. As we see from the non-proton clusters if figures \ref{fig:clster_confmati_filt} and \ref{fig:clster_confmati_raw}  that does not necessarily imply that the physical signals are correspondingly clear. 

%Additionally, a clear disadvantage to the K-means method is that the researcher has to specify the number of clusters in the data themselves. Each experiment then has to be considered in light of possible reaction channels to determine a sensible number of clusters for this approach. 

%The same weakness is present in the MIXAE implementation. Additionally, while it showed better optimal performance than the K-means method, it struggles with inconsistent results. This inconsistency was also notably not evident from the unsupervised training-objectives of the model. These two factors conspire to limit its immediate applicability. 

In summary, we recommend that researchers performing unsupervised track classification begin with an implementation of the VGG16 + K-means approach. For future work, it is worth investigating whether the adaptations of bespoke models like the MIXAE algorithm will allow better performance at no significant cost to consistency.  Especially considering the flexibility to use 3D input representations or auxiliary physical goals in the optimization procedure. 


{\color{orange}[RS Another abrupt end to a paragraph. Could probably use some rounding out.]} 

\appendix  
\section{VGG16}\label{app:vgg}
The choice of the kernel size is based on the fact that a stacked $3 \times 3$ kernel is equivalent to larger kernels in terms of the receptive field of the output. Three $3 \times 3$ kernels with stride $1$ have a $7 \times 7$ receptive field, but the larger kernel has $81\%$ more parameters and only one non-linearity \cite{Simonyan2014}. Stacking the smaller kernels then contributes to a lower computational cost. Additionally, there is a regularizing effect from the lowered number of parameters and increased explanatory power from the additional non-linearities.
\begin{table}
\caption[vgg architectures]{Showing the details of the VGG network architectures. Network D trained on the ImageNet \cite{Russakovsky2015} dataset the network known as  VGG16 and is what we use in this thesis.}\label{tab:vgg}
\includegraphics[width=\textwidth]{plots/vgg_architectures.png}
\end{table}

\section{MIXAE hyper-parameter tuning}

To train the MIXAE clustering algorithm, we use the large simulated dataset with $M=80000$ points, evenly distributed between proton- and carbon-events. The algorithm is trained on a subset of $60000$ of these samples, and we track performance on the remaining $20000$ events. Since there are then only three remaining hyperparameters we choose to perform a coarse grid-search as described in section \ref{sec:hyperparam_search_arch}.

\begin{table}
\centering
\caption{Hyperparameter grid for the MIXAE loss weighting terms. The grid is given as exponents for logarithmic scales.}\label{tab:mixae_loss_weights}
\begin{tabular}{lll}
\toprule
Parameter & Grid & Scale \\
\midrule 
$\theta$ & $[-1,\, 5]$ & Logarithmic \\
$\alpha$ & $[-5,\, -1]$ & Logarithmic \\
$\gamma$ & $[3,\, 5]$ & Logarithmic
\end{tabular}
\end{table}

The grids selected for the search are listed in table \ref{tab:mixae_loss_weights}. The search yielded an optimal configuration with 

\begin{align}
\theta = 10^{-1}, \\
\alpha = 10^{-2}, \\
\gamma = 10^5.
\end{align}

For the full dataset the MIXAE hyperparameters converge to the same values as for the clean data:

\begin{align}
\theta &= 10^{1}, \\
\alpha &= 10^{-1}, \\
\gamma &= 3.162\times 10^3.
\end{align}

Lastly we supply the configuration used for the individual convolutional autoencoder networks in table \ref{tab:mixe_ae_hyperparams}

\begin{table}[H]
\todo{prettify or remove}
\renewcommand*{\arraystretch}{0.5}
\centering
\caption{Hyperparameters selected for the autoencoder components of the MIXAE algorithm, see table \ref{tab:convae_hyperparams} for a full description of the parameters.}\label{tab:mixe_ae_hyperparams}
\setlength{\extrarowheight}{15pt}
\hspace*{-0.5in}
\begin{tabular}{ll}
\toprule
Hyperparameter & Value \\
\midrule
\multicolumn{2}{l}{Convolutional parameters: } \\
\midrule
Number of layers & $4$ \\
Kernels & $[3,\,3,\,3,\,3]$\\
Strides & $[2,\,2,\,2,\,2]$ \\
Filters & $[64,\, 32, \,16, \,8,]$ \\ 
\midrule
\multicolumn{2}{l}{Network parameters: } \\
\midrule
Activation & LReLu \\
Latent dimension & 20  \\
Batchnorm & False \\
\midrule
\multicolumn{2}{l}{Optimizer parameters: } \\
\midrule
$\eta$ & $10^{-3}$ \\
$\beta_1$ & $0.9$ \\
$\beta_2$ & $0.99$ \\
\bottomrule
\end{tabular}
\end{table}


\bibliography{bibliography}
\end{document}
