\documentclass[preprint,12pt]{elsarticle}
\usepackage{amssymb}
\journal{Nuclear Instruments and Methods
in Physics Research Section A: Accelerators, Spectrometers,
Detectors and Associated Equipment}

\begin{document}

\begin{frontmatter}

%% \title{Title\tnoteref{label1}}
%% \tnotetext[label1]{}
%% \author{Name\corref{cor1}\fnref{label2}}
%% \ead{email address}
%% \ead[url]{home page}
%% \fntext[label2]{}
%% \cortext[cor1]{}
%% \address{Address\fnref{label3}}
%% \fntext[label3]{}

\title{Latent Variable Machine Learning Algorithms and Classification of Events}

%% use optional labels to link authors explicitly to addresses:
%% \author[label1,label2]{}
%% \address[label1]{}
%% \address[label2]{}

\author{R.~Solli}
\address{Department of Physics, University of Oslo, POB 1048 Oslo, N-0316 Oslo, Norway}

\author{D.~Bazin}
\address{Department of Physics and Astronomy and Facility for Rare Ion Beams and National Superconducting Cyclotron Facility, Michigan State University, East Lansing, MI 48824, USA}
\author{M.P.~Kuchera}
\address{Physics Department, Davidson College, Davidson, North Carolina, USA}
\author{M.~Hjorth-Jensen}
\address{Department of Physics and Astronomy and Facility for Rare Ion Beams and National Superconducting Cyclotron Facility, Michigan State University, East Lansing, MI 48824, USA}
\address{Department of Physics and Center for Computing in Science Education, University of Oslo, POB 1048 Oslo, N-0316 Oslo, Norway}
\ead{hjensen@frib.msu.edu}
\ead[url]{http://mhjgit.github.io/info/doc/web/}


\begin{abstract}
In this work we introduce the application of convolutional autoencoder neural networks to the analysis of two-dimensional projections of particle tracks from a resonant proton scattering experiment on ${}^{46}$Ar. 
The data we analyze were recorded by an active target time-projection chamber (AT-TPC). Machine learning presents an interesting avenue for researchers operating an AT-TPC, as traditional analysis methods of AT-TPC data are both computationally expensive and fit all particle tracks against the event type of interest. The latter presents a considerable challenge when the space of reactions is not known prior to the analysis. 

We explore the performance of the autoencoder neural networks and a pre-trained VGG16 \cite{Simonyan2014} convolutional neural network on two tasks: a semi-supervised classification task and the unsupervised clustering of particle tracks. On the semi-supervised task, we find that a logistic regression classifier trained on small labelled subsets of the latent space of these models perform very well. On simulated data these classifiers achieve an $f1$ score \cite{Chinchor1992} of $f1>0.95$. The VGG16 latent classifier achieves this result with as few as $N=100$ samples, as does the convolutional autoencoder when trained on the VGG16 representations of the particle tracks. On real data, pre-processed with noise filtering, the same models achieve an $f1>0.7$. For unfiltered real data the models achieve an $f1$ value larger than $0.6$. Both of the previous results were found with the classifiers trained on $N=100$ samples. Furthermore, we found that the autoencoder model reduces the variability in the identification of proton events by $64\%$ from the benchmark logistic regression classifier trained on the VGG16 latent space on real experimental data. 

On the clustering task, we found that a $K$-means algorithm applied to the simulated data in the VGG16 latent space forms almost perfect clusters, with an adjusted rand index \cite{Hubert1985} ($ARI$) $ > 0.8$.  Additionally, the VGG16+K-means approach finds high purity clusters of proton events for real experimental data. We also explore the application of neural networks to clustering by implementing a mixture of autoencoders algorithm. With this model we improved clustering performance on the real experimental data from an $ARI = 0.17$ to an $ARI = 0.40$. However, the neural network clustering suffers from stability issues necessitating further investigations into this approach. 

\end{abstract}

%%Graphical abstract
\begin{graphicalabstract}
%\includegraphics{grabs}
\end{graphicalabstract}

%%Research highlights
\begin{highlights}
\item Research highlight 1
\item Research highlight 2
\end{highlights}

\begin{keyword}
%% keywords here, in the form: keyword \sep keyword

%% PACS codes here, in the form: \PACS code \sep code

%% MSC codes here, in the form: \MSC code \sep code
%% or \MSC[2008] code \sep code (2000 is the default)

\end{keyword}

\end{frontmatter}

\section{Introduction}\label{sec:intro}

\section{Methods}\label{sec:methods}

\section{Results and Discussions}\label{sec:results}

\section{Conclusions and Perspectives}\label{sec:conclusion}

\appendix  



\begin{thebibliography}{00}

%% \bibitem{label}
%% Text of bibliographic item

\bibitem{}

\end{thebibliography}
\end{document}
